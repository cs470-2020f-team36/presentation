% !TeX root=main.tex
\begin{frame}[fragile]
  \frametitle{PIMC: Perfect Information Monte--Carlo}
  \framesubtitle{Run MCTS on Games with Hidden Information}

  As Go-Stop is a game with \gostopemph{hidden information}, the neural net must not obtain the full state of the game.

  $$ \hskip-7pt {\mathop{\text{\Large $g$}}\limits_{\text{game state}}} \xrightarrow[\text{observation}]{\quad f_{i}\quad} {\mathop{\text{\Large $f_i(g)$}}\limits_{\substack{\text{state observable}\\[0.3em]\text{to player $i$}}}} \xrightarrow{\quad f_i^{-1}\quad} {\mathop{\text{\Large $f^{-1}_i(f_i(g))$}}\limits_{\substack{\text{the set of states}\\[0.3em]\text{that seem identical to $i$}}}}\longrightarrow \ {\mathop{\text{\Large $\mathcal D(g)$}}\limits_{\text{a sample}}} $$

  \vspace{1em}

  We sampled some games from $f_i^{-1}(f_i(g))$ randomly and ran MCTS with those games at each step of MCTS.
\end{frame}

\tikzstyle{block} = [draw, fill=white, rectangle, 
    minimum height=3em, text width=0.6em, text=ddarkgreen]
\tikzstyle{smallblock} = [draw, fill=white, rectangle, 
    minimum height=1.3em, text width=0.6em, text=ddarkgreen]
\tikzstyle{sum} = [draw, fill=white, circle, text=ddarkgreen, text width=0.6em, minimum height=1.3em]
\tikzstyle{input} = [coordinate]
\tikzstyle{output} = [coordinate]
\tikzstyle{pinstyle} = [pin edge={to-,thin,black}]

\begin{frame}[fragile]
  \frametitle{Network Layers}
  \framesubtitle{(for now)}

  As Go-Stop boards have \gostopemph{no geometric properties}, it is weird to use convolutional layers in the network. Instead, we replaced them with \gostopemph{fully connected layers}.

  \vspace*{-1em}

  \small
  \begin{tikzpicture}[auto, node distance=5em]
    % We start by placing the blocks
    \node [input, name=input] {};
    \node [block, right=5em of input, label={[yshift=0.3em]$\begin{matrix}
      \text{in: }n_\text{inp}\\[-0.5em]
      \text{out: }n_\text{hid}\\[-0.5em]
    \end{matrix}$}] (fc1) {\!\!FC};
    \node [block, right of=fc1, label={[yshift=0.3em]$\begin{matrix}
      \text{in: }n_\text{hid}\\[-0.5em]
      \text{out: }n_\text{hid}\\[-0.5em]
    \end{matrix}$}] (fc2) {\!\!FC};
    \node [block, right of=fc2, label={[yshift=0.3em]$\begin{matrix}
      \text{in: }n_\text{hid}\\[-0.5em]
      \text{out: }n_\text{hid}\\[-0.5em]
    \end{matrix}$}] (fc7) {\!\!FC};
    \node [block, above right=2em and 4em of fc7, label={[yshift=0.3em]$\begin{matrix}
      \text{in: }n_\text{hid}\\[-0.5em]
      \text{out: }n_\text{hid}\\[-0.5em]
    \end{matrix}$}] (fcp1) {\!\!FC};
    \node [block, right of=fcp1, label={[yshift=0.3em]$\begin{matrix}
      \text{in: }n_\text{hid}\\[-0.5em]
      \text{out: }n_\text{pol}\\[-0.5em]
    \end{matrix}$}] (fcp2) {\!\!FC};
    \node [sum, right of=fcp2, label={[yshift=0.3em]softmax}] (fcps) {};
    \node [output, right=3em of fcps] (p) {};
    \node [block, below=of fcp1, label={[yshift=0.3em]$\begin{matrix}
      \text{in: }n_\text{hid}\\[-0.5em]
      \text{out: }n_\text{hid}\\[-0.5em]
    \end{matrix}$}] (fcv1) {\!\!FC};
    \node [smallblock, below=of fcp2, label={$\begin{matrix}
      \text{in: }n_\text{hid}\\[-0.5em]
      \text{out: }1\\
    \end{matrix}$}] (fcv2) {\!\!FC};
    \node [smallblock, below=2em of fcv2, label={[yshift=-4.5em]$\begin{matrix}
      \text{in: }n_\text{hid}\\[-0.5em]
      \text{out: }1\\
    \end{matrix}$}] (fcv3) {\!\!FC};
    \node [sum, right of=fcv2] (fcvp) {+};
    \node [sum, right of=fcv3, label={[yshift=-3em]$\tanh$}] (fcvt) {};
    \node [output, right=3em of fcvp] (v) {};

    \draw [draw,->] (input) -- node [pos=0.2] {$f_i(g)$} (fc1);
    \draw [draw,->] (fc1) -- (fc2);
    \path (fc2) -- node[auto=false]{\ldots} (fc7);
    \draw [draw,->] (fc7) -- (fcp1) -- (fcp2) -- (fcps) -- node[pos=0.9] {$\pi$} (p);
    \draw [draw,->] (fc7) -- (fcv1) -- (fcv2) -- (fcvp) -- node[pos=0.9] {$v$} (v);
    \draw [draw,->] (fcv1) -- (fcv3) -- (fcvt) -- (fcvp);
  \end{tikzpicture}
\end{frame}