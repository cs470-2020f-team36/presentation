% !TeX root=main.tex
\begin{frame}[fragile]
  \frametitle{Implementation Details}
  \framesubtitle{Hyperparameters (for now)}

  \begin{itemize}
    \item Batch size $B = 32$,
    \item Input dimension $n_\text{inp} = 307$,
    \item Hidden dimension $n_\text{hid} = 256$,
    \item Policy output dimension $n_\text{pol} = 167$,
    \item Learning rate: $0.1 \cdot 0.95^{\max(0, j - 9)}$,
  \end{itemize}
\end{frame}

\begin{frame}[fragile]
  \frametitle{Implementation Details}
  \framesubtitle{Hyperparameters (for now)}

  \begin{itemize}
    \item $\epsilon = 0.25$, $\alpha = 1$ (Dirichlet noise),
    \item $\#$ of search in one MCTS simulation: $96$,
    \item $\#$ of simulations in one network update: $30$,
    \begin{itemize}
      \item This gives about 600 data triples per update.
    \end{itemize}
    \item PUCT constant $c_\text{puct} = 1$,
    \item Temperature parameter $\tau^{(j)} = 1\text{ if }j < 10\text{ else }0.1$,
    \item The size of a sample $\mathcal D(g)$ from $f_i^{-1}(f_i(g))$: $n_\text{samp}^{(j)} = \max(\lfloor 2^{7.5 - j / 8} \rfloor, 4)$,
  \end{itemize}
\end{frame}

\begin{frame}[fragile]
  \frametitle{How to Play}
  \framesubtitle{Special cards}

  \begin{itemize}
    \item The following cards are called \gostopemph{bonus cards}, and they are treated as junk cards whose valued are $2$- or $3$ times of normal junk cards, resp.
    \begin{center}
      \vspace*{-.5em}
      \includegraphics[width=1.5em]{images/cards/bonus2}\quad \includegraphics[width=1.5em]{images/cards/bonus3}
      \vspace*{.5em}
    \end{center}
    \item The values of the following junk cards are double of normal ones'.
    \begin{center}
      \vspace*{-.5em}
      \includegraphics[width=1.5em]{images/cards/J112}\quad \includegraphics[width=1.5em]{images/cards/J120}
      \vspace*{.5em}
    \end{center}
    \item The animal card of September \raisebox{-.8em}{\includegraphics[width=1.5em]{images/cards/A09}} can be used as a double junk card, and it can be moved only once in a game, usually when the player declares Go or Stop.
  \end{itemize}
\end{frame}

\begin{frame}[fragile]
  \frametitle{How to Play}
  \framesubtitle{Throwing a Bomb}

  \begin{itemize}
    \item When there are 3 (or 2) cards of the same month and there are 1 (or 2) card(s) of the same month on the center field, then the player can \gostopemph{throw a bomb}, meaning that throwing all 3 (or 2) cards simultaneously. The player obtains 2 (or 1) chances to flip the top card without throwing a card from their hand at their turn. The player takes those 4 cards of the same month and an additional junk card from the opponent, if possible.
  \end{itemize}
\end{frame}

\begin{frame}[fragile]
  \frametitle{How to Play}
  \framesubtitle{Flipping}

  \begin{itemize}
    \item When the flipped card was a \gostopemph{bonus card}, the player can flip the next top card once more and they takes an additional junk card which the opponent captured.
    \item When a player threw a card without matches on the center field and then the flipped card was matched with the card thrown, it is called a \gostopemph{discard-and-match}. The player takes those cards and an additional junk card from the opponent.
    \item When a player threw a card with one matched card on the center field and then the flipped card was again matched with the card thrown, it is called a \gostopemph{stacking}.
  \end{itemize}
\end{frame}

\begin{frame}[fragile]
  \frametitle{How to Play}
  \framesubtitle{Flipping}

  \begin{itemize}
    \item When a player threw a card with two matched cards on the center field and then the flipped card was again matched with the card thrown, it is called a \gostopemph{\textit{ttadak}}. The player takes those four cards of the same month and an additional junk card from the opponent.
    \item When the center field was cleared after one's turn, it is called \gostopemph{clearing} and the player taken the last turn takes an additional junk card from the opponent.
  \end{itemize}
\end{frame}

\begin{frame}[fragile]
  \frametitle{How to Play}
  \framesubtitle{Go and Stop}

  \begin{itemize}
    \item When either the temporary score of a player (refer to the next slide) is $\ge 7$ and they did not declare `Go' yet, or the temporary score of a player has incrased since the last declaration of `Go' in the other cases, the player can declare `Go' (again) or `Stop.'
    \item When the player declared `Stop,' the game is over and the winner calculates the final score (refer to the next slide.)
    \item When the player declared `Go,' the game continues.
  \end{itemize}
\end{frame}

\begin{frame}[fragile]
  \frametitle{How to Play}
  \framesubtitle{Calculation of Scores}

  \begin{itemize}
    \item There are two kinds of score factors: additive ones and multiplicative ones.
    \item We calculate \gostopemph{the temporary score} by adding all the additive scores.
    \item The temporary score is used to decide whether a player can declare `Go' or `Stop.'
    \item We calculate \gostopemph{the final score} by multiplying the other factors to the temporary score.
    \item Go-Stop is a \gostopemph{zero-sum game}, meaning that the loser's score is the same with the winner's score multiplied by $-1$. 
  \end{itemize}
\end{frame}

\begin{frame}[fragile]
  \frametitle{How to Play}
  \framesubtitle{Additive Score Factors}

  \begin{itemize}
    \item When a player captured $n$ junk cards, its score is $\max(0, n - 9)$. Double junks and triple junks are counted as $2$ and $3$ cards, following the multiplicity of them.
    \item When a player captured $n$ ribbon cards, its score is $\max(0, n - 4)$.
    \item When a player captured $n$ animal cards, its score is $\max(0, n - 4)$.
  \end{itemize}
\end{frame}

\begin{frame}[fragile]
  \frametitle{How to Play}
  \framesubtitle{Additive Score Factors}

  \begin{itemize}
    \item When a player captured $3$ bright cards with the bright of December \raisebox{-.8em}{\includegraphics[width=1.5em]{images/cards/B12}}, it worths $2$ points.
    \item When a player captured $3$ bright cards without the bright of December \raisebox{-.8em}{\includegraphics[width=1.5em]{images/cards/B12}}, it worths $3$ points.
    \item When a player captured $4$ bright cards, it worth $4$ points.
    \item When a player captured $5$ bright cards, it worth $15$ points.
  \end{itemize}
\end{frame}

\begin{frame}[fragile]
  \frametitle{How to Play}
  \framesubtitle{Additive Score Factors}

  \begin{itemize}
    \item When a player captured all blue ribbons \raisebox{-.8em}{\includegraphics[width=1.5em]{images/cards/R06}} \raisebox{-.8em}{\includegraphics[width=1.5em]{images/cards/R09}} \raisebox{-.8em}{\includegraphics[width=1.5em]{images/cards/R10}}, it worths $3$ points.
    \item When a player captured all red ribbons \raisebox{-.8em}{\includegraphics[width=1.5em]{images/cards/R01}} \raisebox{-.8em}{\includegraphics[width=1.5em]{images/cards/R02}} \raisebox{-.8em}{\includegraphics[width=1.5em]{images/cards/R03}}, it worths $3$ points.
    \item When a player captured all plant ribbons \raisebox{-.8em}{\includegraphics[width=1.5em]{images/cards/R04}} \raisebox{-.8em}{\includegraphics[width=1.5em]{images/cards/R05}} \raisebox{-.8em}{\includegraphics[width=1.5em]{images/cards/R07}}, it worths $3$ points.
    \item When a player captured all \textit{five birds} cards \raisebox{-.8em}{\includegraphics[width=1.5em]{images/cards/A02}} \raisebox{-.8em}{\includegraphics[width=1.5em]{images/cards/A04}} \raisebox{-.8em}{\includegraphics[width=1.5em]{images/cards/A08}}, it worths $5$ points.
  \end{itemize}
\end{frame}

\begin{frame}[fragile]
  \frametitle{How to Play}
  \framesubtitle{Additive Score Factors}

  \begin{itemize}
    \item When a player declared `Go' $n$ times, it worths $n$ points.
    \item When there was a \gostopemph{four of a month} in one's hand (all 4 cards of the same month were in one's hand,) then the game ends and the player having the four of a month gets 10 points. If both players had those, then nullify the game.
    \item When a player made \gostopemph{three stackings}, then the player wins with $10$ points (without counting any other additive score factors).
  \end{itemize}
\end{frame}

\begin{frame}[fragile]
  \frametitle{How to Play}
  \framesubtitle{Multiplicative Score Factors}

  \begin{itemize}
    \item When a player declared `Go' $n$ times, the score is multiplied by $2^{\max(0, n - 2)}$. (Together with the additional scores.)
    \item When the winner \textit{shaked} or \textit{threw a bomb} $n$ times, the score is multiplied by $2^n$.
  \end{itemize}
\end{frame}

\begin{frame}[fragile]
  \frametitle{How to Play}
  \framesubtitle{Multiplicative Score Factors}

  \begin{itemize}
    \item When the winner has captured $3$ or more bright cards and the opponent has no bright cards, it is called \gostopemph{bright penalty} and the score is multiplied by 2.
    \item When the winner has captured $7$ or more animal cards, it is called \gostopemph{animal penalty} and the score is multiplied by 2.
    \item When the winner has captured $10$ or more junk cards and the opponent has $\le 7$ junk cards, it is called \gostopemph{junk penalty} and the score is multiplied by 2.
    \item When the loser had declared `Go' at least once before, it is called \gostopemph{go penalty} and the score is multiplied by 2.
  \end{itemize}
\end{frame}